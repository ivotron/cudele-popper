\begin{abstract}
Today's large and highly-parallel workloads are bottlenecked by metadata
services because many processes end up accessing the same shared resoure. In
HPC, state-of-the-art file systems are abandoning POSIX because the
sychronization and serializiation overheads are too costly -- and sometimes
even unnceccessary -- for some of their applications.  While the performance
benefits are plain for these users, other applications that rely on stronger
consistency must be re-written or deployed on a different system.  We present
Cudele, a programmable file system that supports different degrees of
consistency and fault tolerance within the same namespace.  First, we take a
POSIX compliant file system and relax the consistency constraints by
implementing two metadata designs: delayed metadata update merge into the
global namespace and client local views of metadata updates.  Second, we store
the consistency and fault tolerance semantics in inodes so that subtrees within
the same namespace can be optimized for different workloads; the inodes are
programmable so that clients understand how to access the metadata in a
subtree.  Third, we present a theoretical framework for a metadata service and
a working implementation that provides the lowest common denominator needed to
implement a wide range of consistency/fault tolerance semantics that can be
benchmarked all on the same system.
\end{abstract}


